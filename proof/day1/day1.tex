\documentclass{article}
\author{Chirath Nissanka}
\title{Proof Writing: Set Theory (Day 1)}
\date{Nov 7 2023}
\usepackage{amsmath} 
\usepackage{breqn}
\usepackage{amsfonts}
\begin{document}
    \maketitle
    \section{Sets}
    A set is a collection of objects known as elements.
    Sets can be either finite, or infinte.
    \subsection{Finite Sets}
    Examples:
    \[
        \{1,2,3,4,5,6\}   
    \]
    \[
        \{\text{all letters of the alpabet}\}   
    \]
    \subsection{Infinte Sets}
    Examples:
    \[
        A = \{1,3,5,...\}  
    \]
    \[
        B = \{\text{all real numbers}\}   
    \]
Sets are made up of elements 
The relation of "being an element of" is written via:
 \[ a \in A\]
 Means 'a' is an element of 'A'.
 
\section{Warm Up Excercies : Answers}
1.
\begin{dmath}
    \{ x \in \mathbb{Z} \text{ s.t } |7x| < 24 \} =  \\ 
    \{ -3, -2, -1, 0, 1, 2, 3 \} 
\end{dmath}
\newpage
2.
\[
    \begin{split}
        \{ x \in \mathbb{R} \text{ s.t }  7x^2 - x^3 = 12x \}  \\ 
        7x^2 - x^3 - 12x = 0 \\
        x  (7x - x^2 - 12) = 0 \\
        -x  (x^2 - 7x + 12) = 0 \\ 
        -x (x - 3) (x - 4) = 0 \\ 
        x = 0, x = 3, x = 4 \\
        \{ x \in \mathbb{R} \text{ s.t }  7x^2 - x^3 = 12x \}  = \{ 0, 3, 4\}
    \end{split}
\]
3. 
\[
    \begin{split}
        \{ x \in \mathbb{P}(\{1,2,3\}) \text{ s.t } |x| = 2 \}  \\
        \mathbb{P}(\{1,2,3\}) = \{\{1\}, \{2\}, \{3\}, \{1,2\}, \{1,3\}, \{2,3\}, \{1,2,3\}\}\\
        \{ x \in \mathbb{P}(\{1,2,3\}) \text{ s.t } |x| = 2 \}= \{\{1,2\},\{1,3\},\{2,3\}\}
    \end{split}
\]

\end{document}